\documentclass[a4paper,12pt]{article}

\usepackage[utf8]{inputenc}
\usepackage{url}
\usepackage{amsmath,amssymb,amsfonts,bm}
\usepackage{physics,geometry,hyperref,enumitem}
\hypersetup{colorlinks=true, linkcolor=blue, urlcolor=blue, citecolor=blue}

\usepackage[
backend=bibtex,
style=alphabetic,
sorting=ynt
]{biblatex}
\addbibresource{bibliography.bib}


\begin{document}

\title{On the Analytic Continuation of Schr\"odinger Equation}
\date{\today}
\maketitle

\begin{abstract}
	In resent papers \cite{yordanov2024complex} \cite{yang2021extending} the wave function
	was analytically extended to the positions complex plane, arguing that Quantum Mechanics can be understood as a
	Complex Stochastic Optimal Control. In this work we study the analytical continuation of the Schr\"odinger Equation to the 
	complex plane in position and time, how to perform numerical simulations, and the meaning of the results.
\end{abstract}

%%%%%
% Dimensionless Schr

\section{Dimensionless Schr\"odinger Equation}

~~~~We write the Schr\"odinger Equation in dimensionless variables so we can understand better have a better numerical scheme.

First we choose some scales for our system:

\begin{itemize}
\item Length scale $L_0$.
\item Time scale $T_0$.
\item Mass scale $M_0$.
\item Temperature scale $\Theta_0$
\end{itemize}

and a reference energy $E_0$. Then we define the following dimensionless parameters:

\begin{equation}\label{eps_t}
\varepsilon_t = \frac{\hbar}{T_0 E_0 }
\end{equation}

\begin{equation}\label{eps_x}
\varepsilon_x^2 = \frac{\hbar^2}{L_0^2 E_0 M_0}
\end{equation}

\begin{equation}\label{eps_theta}
\varepsilon_{\Theta} = \frac{E_0}{k_B \Theta_0}
\end{equation}

The Schr\"odinger Equation on position basis for single particle is

\begin{equation}\label{schr_dim}
i \hbar \frac{\partial \psi}{\partial t} = - \frac{\hbar^2}{2m}\nabla^2 \psi - V \psi
\end{equation}

in dimensionless \footnote{Note that the wave function on position space also has units of $L_0^{-d/2}$, such that $\int dx^d |\psi|^2 = 1$ is dimensionless.} variables this is:

\begin{equation}\label{schr}
i  \frac{\partial \psi}{\partial t} = - \frac{\varepsilon_x^2}{2 \varepsilon_t} \nabla^2 \psi - \frac{V}{\varepsilon_t} \psi
\end{equation}

Where we have chosen the mass scale $M_0$ the be the mass of the single particle.

The polar decomposition of the wave function

\begin{equation}\label{psi_polar}
\psi = \sqrt{\rho} e^{iS}
\end{equation}


%%%
% continuity and HJ eq

can be used on the Schr\"odinger Equation \ref{schr}, from which the imaginary part gives a continuity equation:

\begin{equation}\label{cont_schr}
\partial_t \rho + \frac{\varepsilon_x^2}{\varepsilon_t} \partial_l \left( \rho \partial^l S \right) = 0
\end{equation}

where the velocity field is given by 

\begin{equation}\label{real_vel_field}
v^l = \frac{\varepsilon_x^2}{\varepsilon_t} \partial^l S
\end{equation}

The real part of the polar decomposition of \ref{schr} gives a Hamilton-Jacobi equation:

\begin{equation}\label{hj_schr}
\partial_t S =   -\frac{\varepsilon_x^2}{2 \varepsilon_t} ( \partial_l S \partial^l S) - \frac{V}{\varepsilon_t} + \frac{\varepsilon_x^2}{2 \varepsilon_t} \frac{\partial_l \partial^l \sqrt{\rho}}{\sqrt{\rho}}
\end{equation}

where the last term is the denominated Quantum potential

As in Classical mechanics we can take the gradient  of \ref{hj_schr} to obtain the time derivative of the velocity field \ref{real_vel_field}

\begin{equation}\label{grad_hj_schr}
\partial_t \left( \frac{\varepsilon_x^2}{ \varepsilon_t}\partial^k S  \right) + \frac{\varepsilon_x^2}{ \varepsilon_t} ( \partial^l S) ( \partial_l \frac{\varepsilon_x^2}{ \varepsilon_t} \partial^k S)  = -  \frac{\varepsilon_x^2}{ \varepsilon_t} \partial^k \left( \frac{V}{\varepsilon_t} - \frac{\varepsilon_x^2}{2 \varepsilon_t} \frac{\partial_l \partial^l \sqrt{\rho}}{\sqrt{\rho}} \right)
\end{equation}

where we can use the material derivative $\frac{d}{dt} = \frac{\partial}{\partial t} + \frac{ \partial x^l}{\partial t } \frac{ \partial}{\partial x^l} $ and write the Newtonian equation:

\begin{equation}\label{newton}
\frac{dv^k}{dt} =  -  \frac{\varepsilon_x^2}{ \varepsilon_t} \partial^kU
\end{equation}

From a numerical point of view Shcr\"odinger Equation \ref{schr} is easier to solve than the polar form \ref{cont_schr} and \ref{hj_schr}. In Classical Mechanics the Quantum potential does not exist, then it is easier to solve Newton equation \ref{newton} for an ensemble of positions and use a Monte Carlo simulation to evaluate mean values: 

\begin{equation}\label{mc_integral}
\int dx O \rho = \frac{1}{n} \sum_{i=1}^{n} O_i  
\end{equation}

where $O_i$ is the observable evaluated on each element of the ensemble of $n$ positions.


%%%
% log schr

\section{Grad-Log-Schr\"odinger Equation}

~~~~Consider 

\begin{equation}\label{log_psi}
\gamma = \log(\psi)
\end{equation}

and use the effective inverse mass parameter

\begin{equation}\label{sigmass}
\sigma_m^2 = \frac{\varepsilon_x^2}{\varepsilon_t} 
\end{equation}

then Schr\"odinger Equation \ref{schr} becomes

\begin{equation}\label{log_schr}
i \partial_t \gamma = - \frac{\sigma_m^2}{2} (\partial_l \partial^l \gamma + \partial_l \gamma \partial^l \gamma) + \frac{V}{\varepsilon_t}
\end{equation}

we call this the Log-Schr\"odinger Equation. Taking the gradient of last equation we obtain


%%%
% grad log schr

\begin{equation}\label{grad_log_schr}
\partial_t \left( \partial^k (-i \sigma_m^2  \gamma)  \right) + \partial^l (-i \sigma_m^2 \gamma )  \partial_l(  \partial^k (-i \sigma_m^2 \gamma) )  = -  \sigma_m^2 \partial^k \left( \frac{V}{\varepsilon_t} - \frac{i}{2} \partial_l \partial^l (-i \sigma_m^2 \gamma) \right)
\end{equation}

we call this the Grad-Log-Schr\"odinger Equation. Using $ \eta^k = -i \sigma_m^2  \partial^k\gamma $ we obtain  a Newtonian like equation

\begin{equation}\label{newton_complex}
\frac{d \eta^k}{dt} = -  \frac{\sigma_m^2}{ \varepsilon_t} \partial^k  V + \frac{i\sigma_m^2}{2 } \partial_l \partial^l \eta^k
\end{equation}

but we have to assume a complex velocity 

\begin{equation}\label{velocity_complex}
 \frac{\partial z^k}{\partial t} = \eta^k
\end{equation}

and the analytical continuation of the potential $V$ \cite{yordanov2024complex}. 

The Newtonian Grad Log Shcr\"odinger equation tells how an ensemble of identical particles moves on the complex plane, where 
the interaction between the particles comes from the velocitypotential $ \frac{i\sigma_m^2}{2 } \partial_l \partial^l \eta^k$. It is clear
that we recover classical mechanics on the complex plane if this interaction is zero, on other cases this is analogous to Bohm's
quantum potential, with the difference that is linear, and does not depend on estimating the density of the particles, but directly uses
the velocity. Before discussing more about this equation, let's study the Log-Schr\"odinger Equation \ref{log_schr}.



%%%
% log schr free particle

\section{Applications}

\subsection{Log-Schr\"odinger Equation on free particle}

~~~~For the free particle we have $V=0$. Let us study the dispersion of a Gaussian wave packet, we begin with the ansatz

\begin{equation}
\gamma = -x^2 f_2(t) - f_0(t)
\end{equation}

using last equation on \ref{log_schr} we get:

\begin{eqnarray}
i \left( -x^2 \dot f_2 - \dot f_0 \right) &=& - \frac{\sigma_m^2}{2} \left( -2 f_2 + 4 x^2 f_2^2 \right)  \nonumber \\
\dot f_2 & = & - 2 i \sigma_m^2 f_2^2 \\
\dot f_0 & = &   i \sigma_m^2 f_2
\end{eqnarray}

that gives the known solutions

\begin{eqnarray}\label{log_free_particle}
f_2(t) & = & \frac{f_2}{1 + 2i \sigma_m^2 f_2 t}  \\
f_0(t) & = & f_0 + \frac{1}{2} \log \left( 1 + 2i \sigma_m^2 f_2 t \right) \\
\psi(x,t) & = & \left( \frac{4 f_2}{2\pi \left( 1 + 2i \sigma_m^2 f_2 t  \right)^{2} } \right)^{1/4} e^{ \frac{-x^2 f_2}{ 
1 + 2i \sigma_m^2 f_2 t} } 
\end{eqnarray}

where we used $ f_0  = - \frac{1}{4} \log \left( \frac{4 f_2}{2 \pi} \right)$ to normalize the wave function. Note that considering the limit 
$f_2->\infty$  on equation \ref{log_free_particle} we have a solution that resembles the free particle propagator
\begin{equation}
\psi(x,t) =   \left( \frac{1}{2\pi \left( i \sigma_m^2 t  \right)^{2} } \right)^{1/4} e^{ \frac{-x^2}{2i \sigma_m^2 t} } 
\end{equation}

An important result of eq \ref{log_free_particle} is the wave packet dispersion relation:
\begin{equation}\label{dispersion_free_particle}
\frac{1}{4\Re(f_2(t))}  = \frac{ (1-2\sigma_m^2\Im(f_2)t)^2 + (2\sigma_m^2\Re(f_2)t)^2}{  4 \Re(f_2)  }
\end{equation}


%%%
% log schr free moving  particle

The "moving" free particle has the ansatz:

\begin{equation}
\gamma = -(x - x_t)^2 f_2(t) - (x-x_t)f_1(t) - f_0(t)
\end{equation}

where $x_t$ is a function of $t$ alone and $\dot x_t$ is a constant. Using last ansatz on \ref{log_schr} we get:
{\tiny
\begin{eqnarray}
i \left( -(x - x_t)^2 \dot f_2 - (x-x_t) \dot f_1 -  \dot f_0 + 2(x - x_t)f_2 \dot x_t +f_1 \dot x_t \right) &=&
 - \frac{\sigma_m^2}{2} \left( -2 f_2 + 4 (x-x_t)^2 f_2^2  + f_1^2   + 4(x-x_t)f_1f_2\right) \nonumber \\
\dot f_2 & = & - 2i \sigma_m^2 f_2^2 \\
\dot f_1 & = & 2 f_2 \dot x_t - 2i \sigma_m^2 f_1f_2 \\
\dot f_0 & = & - \frac{ i \sigma_m^2}{ 2}  (f_1^2 - 2f_2) + f_1\dot x_t
\end{eqnarray}
}

that gives the known solutions

\begin{eqnarray}
f_2(t) & = & \frac{f_2}{1 + 2i \sigma_m^2 f_2 t}  \\
f_1(t) & = &  \frac{-i }{\sigma_m^2} \dot x_t\\
f_0(t) & = & f_0 + \frac{1}{2} \log \left(1 + 2i \sigma_m^2 f_2 t  \right)  - \frac{i}{2\sigma_m^2} \dot x_t^2 t\\
\psi(x,t) & = & \left( \frac{4 f_2}{2\pi \left( 1 + 2i \sigma_m^2 f_2 t \right)^{2} } \right)^{1/4} e^{ \frac{-(x-x_t)^2 f_2}{ 1 + 2i \sigma_m^2 f_2 t} +  \frac{i}{\sigma_m^2} \left(    \dot x_t (x-x_t) + \frac{1}{2} \dot x_t^2 t  \right) } 
\end{eqnarray}

which has the dispersion relation given by \ref{dispersion_free_particle}


%%%
% log schr harmonic

\subsection{Log-Schr\"odinger Equation on harmonic potential}

~~~~For the harmonic potential we have $V=\frac{m \omega'^2}{2} x'^2$, in dimensionless variables is 
$V = \frac{x^2 \omega^2}{2} \frac{\varepsilon_t^2}{\varepsilon_x^2} =\frac{x^2 \omega^2 \varepsilon_t}{2 \sigma_m^2} $.
Which is convenient to rewrite like this:

\begin{equation}\label{har_pot}
V = \frac{ \omega^2 \varepsilon_t}{2 \sigma_m^2}  \left( (x-x_t)^2 + 2 (x-x_t) x_t + x_t^2 \right)
\end{equation}

Considering the ansatz:

\begin{equation}
\gamma = -(x - x_t)^2 f_2(t) - (x-x_t)f_1(t) - f_0(t)
\end{equation}

where $x_t$ is a function of $t$ alone and $\dot x_t$ is not a constant. Using last ansatz on \ref{log_schr} we get:

{\tiny
\begin{eqnarray}\label{log_harmonic}
i \left( -(x - x_t)^2 \dot f_2 - (x-x_t) \dot f_1 -  \dot f_0 + 2(x - x_t)f_2 \dot x_t +f_1 \dot x_t \right) &=&
 - \frac{\sigma_m^2}{2} \left( -2 f_2 + 4 (x-x_t)^2 f_2^2  + f_1^2  + 4(x-x_t)f_1f_2\right) + \nonumber \\
 &&  \frac{\omega^2}{2 \sigma_m^2} \left( (x-x_t)^2 + 2 (x-x_t) x_t + x_t^2 \right)  \nonumber \\
\dot f_2 & = & - 2i \sigma_m^2 f_2^2 + \frac{i\omega^2}{2 \sigma_m^2}  \\
\dot f_1 & = & 2 f_2 \dot x_t - 2i \sigma_m^2 f_1f_2 + x_t\frac{i\omega^2}{\sigma_m^2} \\
\dot f_0 & = & - \frac{ i \sigma_m^2}{ 2}  (f_1^2 - 2f_2) + f_1\dot x_t + \frac{i\omega^2}{2\sigma_m^2} x_t^2
\end{eqnarray}
}

that gives the solution known solution

\begin{eqnarray}
\ddot  x_t &=& - \omega^2 x_t \\
f_2(t) &=& \frac{\omega}{2\sigma_m^2} \\
f_1(t)  &=& -i \frac{\dot x_t}{\sigma_m^2} \\
f_0(t) &=& i\frac{\omega t}{2}- i \int dt \left(  \frac{\dot x_t^2}{2\sigma_m^2} - \frac{\omega^2 x_t^2}{2 \sigma_m^2}\right)
\end{eqnarray}
that is the coherent state solution when $x_t(t) \neq 0$ and the ground state when $x_t = 0$


An alternative solution of eq \ref{log_harmonic} is given by:

\begin{eqnarray}\label{log_harmonic_sol}
\ddot  x_t &=& - \omega^2 x_t \\
f_2(t) &=& i \frac{\omega}{2\sigma_m^2} \tan \left(  \omega t  + \phi_0 \right) \\
f_1(t)  &=& -i \frac{\dot x_t}{\sigma_m^2} \\
f_0(t) &=& \frac{1}{2}\log \left( \cos(\omega t + \phi_0) \right) - i \int dt \left(  \frac{\dot x_t^2}{2\sigma_m^2} - \frac{\omega^2 x_t^2}{2 \sigma_m^2}\right)
\end{eqnarray}

making $x_t(t) = 0$ and $\phi_0 = \pi/2$ we have the solution that resembles the harmonic propagator. 

Considering $\phi_0$ as a complex number, then dispersion relation of eq \ref{log_harmonic_sol} is

\begin{equation}\label{dispersion_harmonic}
\frac{1}{4\Re(f_2(t))} = \frac{ \sigma_m^2 }{2 \omega \tanh(Im(\phi_0))}  (\cos^2(\omega t + \Re(\phi_0)) + \tanh^2(\Im(\phi_0)) \sin^2(\omega t + \Re(\phi_0)))
\end{equation}



%%%
% newton grad log schr free particle

\subsection{Newton Grad Log Shcr\"odinger Equation on free particle}

~~~~For the free particle in one dimension we have

\begin{equation}
\frac{d \eta}{dt} =  \frac{i\sigma_m^2}{2 } \partial_l \partial^l \eta
\end{equation}

and then the cinematic equation for z is

\begin{equation}
z  =  z_0 + \int_0^t dt \eta(t)
\end{equation}

If we start from the initial condition

\begin{equation}
\gamma(z,0) = -z^2 f_2 - f_0
\end{equation} 

then for each initial position $z_0$ of the ensemble of particles  on the complex plane, we have the initial velocity

\begin{equation}
\eta_0 = 2 i \sigma_m^2 f_2 z_0
\end{equation}

even more, since $\partial_l \partial^l \eta = 0$, then $\eta_0$ is constant, then the cinematic equation becomes

\begin{equation}
z(t) = z_0(1+2i\sigma_m^2f_2t)
\end{equation}

in this case the dispersion relation is given by 

\begin{eqnarray}
\langle z(t) \bar z(t) \rangle -  \langle z(t) \rangle \langle \bar z(t) \rangle  & = & 
(\langle z_0 \bar z_0 \rangle -  \langle z_0 \rangle \langle \bar z_0 \rangle)|1+2i\sigma_m^2f_2t|^2 \nonumber \\
& = & \sigma_0^2 \left( (1 - 2\sigma_m^2\Im(f_2)t)^2 + (2\sigma_m^2\Re(f_2)t)^2 \right) \label{dispersion_free_particle_2}
\end{eqnarray}

which is equal to the dispersion relation \ref{dispersion_free_particle}


The moving free particle has the initial condition 
\begin{equation}
\gamma(z,0) = -(z-a_0)^2 f_2 - (z-a_0) f_1 - f_0
\end{equation} 

where we use $a_0$ to indicate a constant complex number, that indicates the center of the distribution, which is different from $z_0$, that is the initial condition of each particle on the ensemble.

Then for each initial position $z_0$ of the ensemble of particles  on the complex plane, we have the initial velocity

\begin{equation}
\eta_0 = 2 i \sigma_m^2 f_2 (z_0 - a_0) + i \sigma_m^2 f_1
\end{equation}

again this is constant, then the cinematic equation becomes

\begin{equation}
z(t) = z_0(1+2i\sigma_m^2f_2t) + i\sigma_m^2(f_1 - 2f_2a_0)t
\end{equation}

even though the mean value changes, we have the same dispersion \ref{dispersion_free_particle_2}.


%%%
% newton grad log schr harmonic potential

\subsection{Newton Grad Log Shcr\"odinger Equation on harmonic potential}

~~~~For the particle in one dimensional harmonic potential we have

\begin{equation}
\frac{d \eta}{dt} = - \omega^2 z + \frac{i\sigma_m^2}{2 } \partial_l \partial^l \eta
\end{equation}

using the initial condition 
\begin{equation}
\gamma(z,0) = -(z-a_0)^2 f_2 - (z-a_0) f_1 - f_0
\end{equation} 

then for each initial position $z_0$ of the ensemble of particles  on the complex plane, we have the initial velocity

\begin{equation}
\eta_0 = 2 i \sigma_m^2 f_2 (z_0 - a_0) + i \sigma_m^2 f_1
\end{equation}

since again we have $\frac{i\sigma_m^2}{2 } \partial_l \partial^l \eta = 0$ then the cinematic solution is given by

\begin{eqnarray}
z(t) & = & z_0 \cos(wt) + \frac{\eta_0}{w} \sin(wt) \nonumber \\
& = & z_0 \left(  \cos(\omega t) + \frac{2i\sigma_m^2f_2}{\omega} \sin(\omega t)\right) + \frac{i \sigma_m^2 (f_1 -2f_2a_0)}{\omega}\sin(\omega t)
\end{eqnarray}

and dispersion relation

\begin{equation}
\sigma^2(t) = \sigma_0^2\left( \left( \cos(\omega t) - \frac{2\sigma_m^2\Im(f_2)}{\omega} \sin(\omega t) \right)^2 + \left( \frac{2\sigma_m^2\Re(f_2)}{\omega} \sin(\omega t)  \right)^2 \right)
\end{equation}
 
which is equal to the relation \ref{dispersion_harmonic} once that we recognize 

\begin{eqnarray}
\Re(f_2) & = & \frac{\omega}{2\sigma_m^2} \frac{\cosh(\Im(\phi_0)) \sinh(\Im(\phi_0))}{\cos^2(\Re(\phi_0)) + \sinh^2(\Im(\phi_0))}  \\
\Im(f_2) & = & \frac{\omega}{2\sigma_m^2} \frac{\cos(\Re(\phi_0)) \sin(\Re(\phi_0))}{\cos^2(\Re(\phi_0)) + \sinh^2(\Im(\phi_0))} 
\end{eqnarray}

\printbibliography

\end{document}
